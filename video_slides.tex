\documentclass[12pt]{beamer}
\usetheme{Warsaw}
\usepackage[utf8]{inputenc}
\usepackage{amsmath}
\usepackage{amsfonts}
\usepackage{amssymb}
\usepackage{graphicx}
\setbeamersize{text margin left=8pt,text margin right=8pt}

\author{Agustín Curto}
\title{Turing vence a von Neumann}
\setbeamercovered{transparent}
\setbeamertemplate{navigation symbols}{}
\logo{}
\institute{FaMAF}
\date{2017}

\newcommand{\SIGMA}{\Sigma^{\ast}}
\newcommand{\PN}{\par\noindent}

\begin{document}

\begin{frame}
	\titlepage
\end{frame}

\begin{frame}
	\textbf{Notación}: dados $x_{1}, \dotsc, x_{n} \in \omega$ y $\alpha_{1}, \dotsc,\alpha_{m} \in \SIGMA$, con $n, m \in
	\omega $, usaremos:
	\begin{equation*}
		\lVert x_{1}, \dotsc, x_{n}, \alpha_{1}, \dotsc, \alpha_{m} \rVert
	\end{equation*}

	\PN para denotar el estado
	\begin{equation*}
		\left((x_{1}, \dotsc, x_{n}, 0, \dotsc), (\alpha_{1}, \dotsc, \alpha_{m}, \varepsilon, \dotsc)\right)
	\end{equation*}

	\PN Notese que por ejemplo:
	\begin{eqnarray*}
	\lVert x \rVert &=& \left((x, 0, \dotsc), (\varepsilon, \dotsc)\right) \text{Para } n = 1, m = 0 \\
	\lVert \Diamond \rVert &=& \left((0, \dotsc), (\varepsilon, \dotsc)\right) \text{Para } n = m = 0
	\end{eqnarray*}

	\PN Ademas es claro que:
	\[
		\lVert x_{1}, \dotsc, x_{n}, \alpha_{1}, \dotsc, \alpha_{m} \rVert = \lVert x_{1}, \dotsc, x_{n},
		\overset{i}{\overbrace{0, \dotsc, 0}}, \alpha_{1}, \dotsc, \alpha_{m}, \overset{j}{\overbrace{\varepsilon, \dotsc,
		\varepsilon}} \rVert
	\]
	cualesquiera sean $i, j \in \omega$.
\end{frame}

\end{document}
