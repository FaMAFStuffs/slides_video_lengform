\begin{frame}
  Para armar la maquina que simulara a $\mathcal{P}$ hacemos lo siguiente.
Primero unimos las maquinas $M_{1},...,M_{n}$ de la siguiente manera

\bigskip

\bigskip

\bigskip

\bigskip

Nueva Figura 6

\bigskip

\bigskip

\bigskip

\bigskip

\noindent Luego para cada $i$ tal que $Bas(I_{i})$ es de la forma $\alpha
\mathrm{GOTO}\;\mathrm{L}\bar{m}$, ligamos con una flecha de la forma%
\begin{equation*}
\underrightarrow{\;\;\;\;\;\;B,B,K\;\;\;\;\;\;}
\end{equation*}%
el estado final $q_{si}$ de la $M_{i}$ con el estado inicial de la $M_{h}$,
donde $h$ es tal que $I_{h}$ es la primer instruccion que tiene label $%
\mathrm{L}\bar{m}$.
\end{frame}
