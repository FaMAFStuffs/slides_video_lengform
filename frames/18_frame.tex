\begin{frame}
  \begin{block}
    \PN Supongamos que $\mathcal{P} = I_{1}, \dotsc, I_{n}$. Para cada $i = 1, \dotsc, n$, llamaremos $M_{i}$ a la
    máquina que simulará el efecto que produce la instrucción $I_{i}$, es decir tomemos:
    \begin{enumerate}
      \item[-] $M_{i}=M_{j,k}^{+}$, si $Bas(I_{i}) = \mathrm{N}\bar{j} \leftarrow \mathrm{N}\bar{j}+1$

      \item[-] $M_{i}=M_{j,k}^{\dot{-}}$, si $Bas(I_{i}) = \mathrm{N}\bar{j} \leftarrow \mathrm{N}\bar{j}\dot{-}1$

      \item[-] $M_{i}=M_{j,k}^{a}$, si $Bas(I_{i})=\mathrm{P}\bar{j} \leftarrow \mathrm{P}\bar{j}.a$

      \item[-] $M_{i}=M_{j,k}^{\curvearrowright }$, si $Bas(I_{i}) = \mathrm{P}\bar{j} \leftarrow \
      ^{\curvearrowright}\mathrm{P}\bar{j}$

      \item[-] $M_{i}=M_{j\leftarrow m}^{\#,k}$, si $Bas(I_{i}) = \mathrm{N}\bar{j} \leftarrow \mathrm{N}\bar{m}$

      \item[-] $M_{i}=M_{j\leftarrow m}^{\ast ,k}$, si $Bas(I_{i}) = \mathrm{P}\bar{j} \leftarrow \mathrm{P}\bar{m}$

      \item[-] $M_{i}=M_{j\leftarrow 0}^{k}$, si $Bas(I_{i}) = \mathrm{N}\bar{j} \leftarrow 0$

      \item[-] $M_{i}=M_{j\leftarrow \varepsilon }^{k}$, si $Bas(I_{i}) = \mathrm{P}\bar{j} \leftarrow \varepsilon$

      \item[-] $M_{i}=M_{\mathrm{SKIP}}$, si $Bas(I_{i}) = \mathrm{SKIP}$

      \item[-] $M_{i}=IF_{j,k}$, si $Bas(I_{i})=\mathrm{IF}\;\mathrm{N}\bar{j} \neq 0 \ \mathrm{GOTO}\;
      \mathrm{L}\bar{m}$, para algún $m$

      \item[-] $M_{i}=IF_{j,k}^{a}$, si $Bas(I_{i}) = \mathrm{IF}\;\mathrm{P}\bar{j} \;\mathrm{BEGINS}\;a\;\mathrm{GOTO}
      \;\mathrm{L}\bar{m}$, para algún $m$
    \end{enumerate}
  \end{block}
\end{frame}
