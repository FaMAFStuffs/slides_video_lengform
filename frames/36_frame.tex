\begin{frame}
  \begin{block}{}
    \PN Pero entonces el Lema \ref{simulacion} nos dice que

    \begin{enumerate}
    \item[(***)] $\left\lfloor q_{0}B\shortmid ^{x_{1}}B...B\shortmid
    ^{x_{n}}B^{k-n}B\alpha _{1}B...B\alpha _{m}B\right\rfloor \overset{\ast }{%
    \underset{M_{sim}}{\vdash }}\left\lfloor q_{f}B^{k+1}f(\vec{x},\vec{\alpha}%
    )\right\rfloor $
    \end{enumerate}

    Como ya lo vimos, si hacemos funcionar a $M$ desde $\left\lfloor
    q_{0}B\shortmid ^{x_{1}}B...B\shortmid ^{x_{n}}B\alpha _{1}B...B\alpha
    _{m}B\right\rfloor $, llegaremos (via la copia de $M_{1}$ dentro de $M$)
    indefectiblemente a la siguiente descripcion instantanea%
    \begin{equation*}
    \left\lfloor q_{2}B\shortmid ^{x_{1}}B...B\shortmid ^{x_{n}}B^{k-n}B\alpha
    _{1}B...B\alpha _{m}B\right\rfloor
    \end{equation*}%
    Luego (***) nos dice que, via la copia de $M_{sim}$ dentro de $M$,
    llegaremos a $\left\lfloor q_{3}B^{k+1}f(\vec{x},\vec{\alpha})\right\rfloor $
    e inmediatamente a $\left\lfloor q_{4}B^{k+1}f(\vec{x},\vec{\alpha}%
    )\right\rfloor $. Finalmente, via la copia de $M_{2}$ dentro de $M$,
    llegaremos a $\left\lfloor q_{5}Bf(\vec{x},\vec{\alpha})\right\rfloor $, lo
    cual termina de demostrar que $M$ computa a $f$
  \end{block}
\end{frame}
