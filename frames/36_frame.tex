\begin{frame}
  \begin{block}{}
    \begin{itemize}
      \item La máquina $M$ se detiene en $\left\lfloor q_{5} B f(\vec{x},\vec{\alpha}) \right\rfloor$, ya que $q_{5}$ es
        el estado final de una copia de $M_{2}$ y por lo tanto no sale ninguna flecha desde él.

      \item Ya que $\mathcal{P}$ computa a $f$ y tiene la propiedad (2) del Lema 2, tenemos que $\mathcal{P}$ termina
        partiendo de $\lVert x_{1}, \dotsc, x_{n}, \alpha_{1}, \dotsc, \alpha_{m} \rVert$ y llega al estado $\lVert
        f(\vec{x},\vec{\alpha}) \rVert$, o lo que es lo mismo, $\mathcal{P}$ termina partiendo de
        \begin{equation*}
          \lVert x_{1}, \dotsc, x_{n}, \overset{k-n}{\overbrace{0,\dotsc,0}}, \alpha_{1}, \dotsc, \alpha_{m},
          \overset{k-m}{\overbrace{\varepsilon,\dotsc,\varepsilon}} \rVert
        \end{equation*}
        \PN y llega al estado
        \begin{equation*}
          \lVert \overset{k}{\overbrace{0,\dotsc,0}}, f(\vec{x}, \vec{\alpha}), \overset{k-1}{\overbrace{\varepsilon,
          \dotsc,\varepsilon}} \rVert
        \end{equation*}

        \PN Pero entonces el Lema 1 nos dice que:
        \begin{center}
          $\left\lfloor q_{0}B\shortmid ^{x_{1}}B\dotsc B\shortmid
          ^{x_{n}}B^{k-n}B\alpha _{1}B\dotsc B\alpha _{m}B\right\rfloor \overset{\ast }{%
          \underset{M_{sim}}{\vdash }}\left\lfloor q_{f}B^{k+1}f(\vec{x},\vec{\alpha}%
          )\right\rfloor $ \textcolor{red}{(***)}
        \end{center}
    \end{itemize}
  \end{block}
\end{frame}
