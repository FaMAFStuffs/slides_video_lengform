\begin{frame}
	A lo que queda entre dos blancos consecutivos (es decir que no hay ningun
	blanco entre ellos) lo llamaremos "bloque", por ejemplo en la cinta de
	arriba tenemos que los primeros 12 bloques son%
	\begin{equation*}
	\shortmid \shortmid \shortmid \ \ \ \shortmid \shortmid \ \ \ \shortmid
	\shortmid \shortmid \shortmid \shortmid \ \ \ \ \varepsilon \ \ \ \
	\shortmid \shortmid \shortmid \shortmid \ \ \ \shortmid \shortmid \ \ \ \&\
	\ \ \ \ \&\&\ \ \ \ \ \varepsilon \ \ \ \ \ \#\&\ \ \ \ \ \#\ \ \ \ \ \#\#\#
	\end{equation*}%
	y despues los bloques siguientes (que son infinitos ya que la cinta es
	infinita hacia la derecha) son todos iguales a $\varepsilon $.

	Una observacion importante es que esta forma de representacion de estados en
	la cinta depende del $k$ elejido, es decir si tomaramos otro $k$, por
	ejemplo $k=9$, entonces el estado anterior se representaria de otra forma en
	la cinta.
\end{frame}
