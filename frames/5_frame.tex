\begin{frame}
	\begin{block}{Definición}
		\PN A lo que queda entre dos blancos consecutivos, es decir, que no hay ningún blanco entre ellos, lo llamaremos
		\textit{bloque}.
	\end{block}

	\vspace{3mm}
	\PN \underline{\textbf{Ejemplo:}} en la cinta de arriba tenemos que los primeros 12 bloques son
	\begin{equation*}
		\shortmid \shortmid \shortmid \ \ \ \shortmid \shortmid \ \ \ \shortmid \shortmid \shortmid \shortmid \shortmid \ \
		\ \ \varepsilon \ \ \ \ \shortmid \shortmid \shortmid \shortmid \ \ \ \shortmid \shortmid \ \ \ \& \ \ \ \ \ \&\& \
		\ \ \ \ \varepsilon \ \ \ \ \ \#\& \ \ \ \ \ \# \ \ \ \ \ \#\#\#
	\end{equation*}
	\PN luego, los bloques siguientes, son todos iguales a $\varepsilon$.

	\begin{alertblock}{Observación}
		\PN Es que esta forma de representación de estados en la cinta depende del $k$ elegido, es decir, si tomaramos otro
		$k$, por ejemplo $k=9$, entonces el estado anterior se representaría de otra forma en la cinta.
	\end{alertblock}
\end{frame}
