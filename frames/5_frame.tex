\begin{frame}
	\textbf{Definición}: A lo que queda entre dos blancos consecutivos, es decir que no hay ningún blanco entre ellos, lo
	llamaremos \textit{bloque}.

	\vspace{3mm}
	\PN \underline{Ejemplo}: en la cinta de arriba tenemos que los primeros 12 bloques son
	\begin{equation*}
		\shortmid \shortmid \shortmid \ \ \ \shortmid \shortmid \ \ \ \shortmid \shortmid \shortmid \shortmid \shortmid \ \
		\ \ \varepsilon \ \ \ \ \shortmid \shortmid \shortmid \shortmid \ \ \ \shortmid \shortmid \ \ \ \& \ \ \ \ \ \&\& \
		\ \ \ \ \varepsilon \ \ \ \ \ \#\& \ \ \ \ \ \# \ \ \ \ \ \#\#\#
	\end{equation*}

	\PN luego, los bloques siguientes, son todos iguales a $\varepsilon$.

	\vspace{3mm}
	\PN \underline{Observación}: es que esta forma de representación de estados en la cinta depende del $k$ elejido, es
	decir, si tomaramos otro $k$, por ejemplo $k=9$, entonces el estado anterior se representaría de otra forma en la
	cinta.
\end{frame}
