\begin{frame}
	\PN Necesitaremos tener alguna manera de representar en la cinta los diferentes estados por los cuales se va pasando,
	a medida que corremos a $\mathcal{P}$. Esto lo haremos de la siguiente forma, al estado
	\begin{equation*}
		\lVert x_{1}, \dotsc, x_{k}, \alpha_{1}, \dotsc, \alpha_{k} \rVert
	\end{equation*}

	\PN lo representaremos en la cinta de la siguiente manera
	\begin{equation*}
		B \shortmid^{x_{1}} B \dotsc B \shortmid^{x_{k}} B \alpha_{1} B \dotsc B \alpha_{k} BBBB \dotsc
	\end{equation*}

	\PN \underline{\textbf{Ejemplo:}} consideremos el programa $\mathcal{P}$ mostrado recién y fijemos $k=6$, entonces al
	estado
	\begin{equation*}
		\lVert 3, 2, 5, 0, 4, 2, \&, \&\&, \varepsilon, \#\&, \#, \#\#\# \rVert
	\end{equation*}

	\PN lo representaremos en la cinta de la siguiente manera
	\begin{equation*}
		B\mathrm{\shortmid \shortmid \shortmid} B \mathrm{\shortmid \shortmid} B\mathrm{\shortmid \shortmid \shortmid
		\shortmid \shortmid} BB \mathrm{\shortmid \shortmid \shortmid \shortmid} B \mathrm{\shortmid \shortmid} B \& B \&\&
		BB \#\& B \# B \#\#\# BBBBB \dotsc
	\end{equation*}
\end{frame}
