\begin{frame}
  Notese que%
  \begin{equation*}
  \begin{array}{lcr}
  \alpha B\beta _{1}B\beta _{2}B...B\beta _{j}B\gamma  & \overset{\ast }{%
  \vdash } & \alpha B\beta _{1}B\beta _{2}B...B\beta _{j}B\gamma  \\
  \ \ \uparrow  &  & \uparrow \ \  \\
  \ \ q_{0} &  & q_{f}\ \
  \end{array}%
  \end{equation*}%
  siempre que $\alpha ,\gamma \in \Gamma ^{\ast }$, $\beta _{1},...,\beta
  _{j}\in (\Gamma -\{B\})^{\ast }$. Es decir la maquina $D_{j}$ lo unico que
  hace es mover el cabezal desde el blanco de la izquierda de un bloque
  determinado, exactamente $j$ bloques a la derecha

  Analogamente $I_{j}$ sera una maquina que desplaza el cabezal $j$ bloques a
  la izquierda del blanco que esta escaneando. Es decir $I_{j}$ cumplira que%
  \begin{equation*}
  \begin{array}{lcr}
  \alpha B\beta _{j}B...B\beta _{2}B\beta _{1}B\gamma  & \overset{\ast }{%
  \vdash } & \alpha B\beta _{j}B...B\beta _{2}B\beta _{1}B\gamma  \\
  \ \ \ \ \ \ \ \ \ \ \ \ \ \ \ \ \ \ \ \ \ \ \ \ \ \ \ \ \uparrow  &  &
  \uparrow \ \ \ \ \ \ \ \ \ \ \ \ \ \ \ \ \ \ \ \ \ \ \ \ \ \ \ \  \\
  \ \ \ \ \ \ \ \ \ \ \ \ \ \ \ \ \ \ \ \ \ \ \ \ \ \ \ \ \ q_{0} &  & q_{f}\
  \ \ \ \ \ \ \ \ \ \ \ \ \ \ \ \ \ \ \ \ \ \ \ \ \ \ \
  \end{array}%
  \end{equation*}%
  siempre que $\alpha ,\gamma \in \Gamma ^{\ast }$, $\beta _{1},...,\beta
  _{j}\in (\Gamma -\{B\})^{\ast }$. Dejamos al lector la manufactura de esta
  maquina.
\end{frame}
