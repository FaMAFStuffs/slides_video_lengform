\begin{frame}
  \begin{alertblock}{Probaremos}
    \PN El paradigma computacional de Turing es por lo menos tan expresivo como el paradigma imperativo dado por el
    lenguaje $\mathcal{S}^{\Sigma}$, es decir, probaremos que toda función $\Sigma$-computable es $\Sigma$-Turing
    computable.
  \end{alertblock}

  \PN Antes un lema:
  \begin{lemma}
    \PN Si $f: D_{f} \subseteq \omega^{n}\times \Sigma^{\ast m} \rightarrow \SIGMA$ es $\Sigma$-computable, entonces
    existe un programa $\mathcal{Q}$, el cual computa a $f$ y cumple con las siguientes propiedades:
    \begin{enumerate}[1)]
      \item En $\mathcal{Q}$ no hay instrucciones de la forma $\mathrm{GOTO} \ \mathrm{L}\bar{\imath}$, ni de la forma
      $\mathrm{L}\bar{j} \ \mathrm{GOTO} \ \mathrm{L}\bar{\imath}$.

      \item Cuando $\mathcal{Q}$ termina partiendo de un estado cualquiera dado, el estado alcanzado es tal que las
      variables numéricas tienen todas el valor $0$ y las alfabéticas tienen, todas exepto $\mathrm{P}1$, el valor
      $\varepsilon$.
    \end{enumerate}
  \end{lemma}
\end{frame}
