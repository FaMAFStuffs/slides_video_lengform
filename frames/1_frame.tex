\begin{frame}
	\frametitle{Introduciendo notación}

	\begin{block}{Notación}
		\PN Dados $x_{1}, \dotsc, x_{n} \in \omega$ y $\alpha_{1}, \dotsc,\alpha_{m} \in \SIGMA$, con $n, m \in \omega $,
		usaremos:
		\begin{equation*}
			\lVert x_{1}, \dotsc, x_{n}, \alpha_{1}, \dotsc, \alpha_{m} \rVert
		\end{equation*}

		\PN para denotar el estado
		\begin{equation*}
			\left((x_{1}, \dotsc, x_{n}, 0, \dotsc), (\alpha_{1}, \dotsc, \alpha_{m}, \varepsilon, \dotsc)\right)
		\end{equation*}
	\end{block}

	\PN Nótese que por ejemplo:
	\begin{eqnarray*}
		\lVert x \rVert &=& \left((x, 0, \dotsc), (\varepsilon, \dotsc)\right) \qquad \text{Para } n = 1, m = 0 \\
		\lVert \Diamond \rVert &=& \left((0, \dotsc), (\varepsilon, \dotsc)\right) \qquad \ \ \; \text{Para } n = m = 0
	\end{eqnarray*}

	\PN Además es claro que:
	\[
		\lVert x_{1}, \dotsc, x_{n}, \alpha_{1}, \dotsc, \alpha_{m} \rVert = \lVert x_{1}, \dotsc, x_{n},
		\overset{i}{\overbrace{0, \dotsc, 0}}, \alpha_{1}, \dotsc, \alpha_{m}, \overset{j}{\overbrace{\varepsilon, \dotsc,
		\varepsilon}} \rVert
	\]
	\PN cualesquiera sean $i, j \in \omega$.
\end{frame}
