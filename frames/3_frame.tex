\begin{frame}
	\PN Sea $\mathcal{P}$ un programa y sea $k$ fijo y $k \leq N(\mathcal{P})$. Describiremos como puede construirse una
	máquina de Turing la cual simulará a $\mathcal{P}$. La construcción de la máquina simuladora dependerá de
	$\mathcal{P}$ y de $k$.

	\PN Nótese que cuando $\mathcal{P}$ se corre desde algún estado de la forma
	\begin{equation*}
		\lVert x_{1}, \dotsc, x_{k}, \alpha_{1}, \dotsc, \alpha_{k} \rVert
	\end{equation*}

	\PN los sucesivos estados por los que va pasando son todos de la forma
	\begin{equation*}
		\lVert y_{1}, \dotsc, y_{k}, \beta_{1}, \dotsc, \beta_{k} \rVert
	\end{equation*}
	\PN es decir, en todos ellos las variables con índice mayor que $k$ valen $0$ o $\varepsilon$.

	\PN La razón es simple, ya que en $\mathcal{P}$ no figuran las variables
	\begin{eqnarray*}
		&&\mathrm{N}\overline{k+1}, \ \mathrm{N}\overline{k+2}, \dotsc \\
		&&\mathrm{P}\overline{k+1}, \ \mathrm{P}\overline{k+2}, \dotsc
	\end{eqnarray*}
	\PN estas variables quedan con valores $0$ y $\varepsilon$, respectivamente a lo largo de toda la computación.
\end{frame}
