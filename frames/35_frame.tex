\begin{frame}
  \begin{block}{}
    \PN Luego entonces (**) nos dice que al seguir trabajando $M$ (ahora via la
    copia de $M_{sim}$ dentro de $M$), la maquina $M$ nunca terminara.

    Para terminar de ver que $M$ computa a $f$, tomemos $(\vec{x},\vec{\alpha}%
    )\in D_{f}$ y veamos que%
    \begin{equation*}
    \left\lfloor q_{0}B\shortmid ^{x_{1}}B...B\shortmid ^{x_{n}}B\alpha
    _{1}B...B\alpha _{m}B\right\rfloor \overset{\ast }{\underset{M}{\vdash }}%
    \left\lfloor q_{5}Bf(\vec{x},\vec{\alpha})\right\rfloor
    \end{equation*}%
    y que la maquina $M$ se detiene en $\left\lfloor q_{5}Bf(\vec{x},\vec{\alpha}%
    )\right\rfloor $. La maquina $M$ se detiene en $\left\lfloor q_{5}Bf(\vec{x},%
    \vec{\alpha})\right\rfloor $ ya que $q_{5}$ es el estado final de una copia
    de $M_{2}$ y por lo tanto no sale ninguna flecha desde el. Ya que $\mathcal{P%
    }$ computa a $f$ y tiene la propiedad (2) del Lema \ref{sinGOTO}, tenemos
    que $\mathcal{P}$ termina partiendo de $\left\Vert x_{1},...,x_{n},\alpha
    _{1},...,\alpha _{m}\right\Vert $ y llega al estado $\left\Vert f(\vec{x},%
    \vec{\alpha})\right\Vert $, o lo que es lo mismo, $\mathcal{P}$ termina
    partiendo de%
    \begin{equation*}
    \left\Vert x_{1},...,x_{n},\overset{k-n}{\overbrace{0,...,0}},\alpha
    _{1},...,\alpha _{m},\overset{k-m}{\overbrace{\varepsilon ,...,\varepsilon }}%
    \right\Vert
    \end{equation*}%
    y llega al estado%
    \begin{equation*}
    \left\Vert \overset{k}{\overbrace{0,...,0}},f(\vec{x},\vec{\alpha}),\overset{%
    k-1}{\overbrace{\varepsilon ,...,\varepsilon }}\right\Vert
    \end{equation*}
  \end{block}
\end{frame}
