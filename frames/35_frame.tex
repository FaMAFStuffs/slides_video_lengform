\begin{frame}
  \begin{block}{}
    \PN Ahora, nótese que si hacemos funcionar a $M$ desde la descripción instantánea dada en \textcolor{red}{(*)},
    llegaremos indefectiblemente a la siguiente descripción instantánea:
    \begin{equation*}
      \left\lfloor q_{2} B \shortmid^{x_{1}} B \dotsc B \shortmid^{x_{n}} B^{k-n}B \alpha_{1} B \dotsc B \alpha_{m} B
      \right\rfloor
    \end{equation*}

    \PN entonces \textcolor{red}{(**)} nos dice que al seguir trabajando $M$, la máquina $M$ nunca terminará.
  \end{block}

  \begin{block}{}
    \PN Para terminar de ver que $M$ computa a $f$, tomemos $(\vec{x},\vec{\alpha}) \in D_{f}$ y veamos que
    \begin{equation*}
      \left\lfloor q_{0}B\shortmid ^{x_{1}}B...B\shortmid ^{x_{n}}B\alpha
      _{1}B...B\alpha _{m}B\right\rfloor \overset{\ast }{\underset{M}{\vdash }}%
      \left\lfloor q_{5}Bf(\vec{x},\vec{\alpha})\right\rfloor
    \end{equation*}
    \PN y que la máquina $M$ se detiene en $\left\lfloor q_{5} B f(\vec{x},\vec{\alpha})\right\rfloor$.
  \end{block}
\end{frame}
