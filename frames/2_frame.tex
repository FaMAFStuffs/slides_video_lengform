\begin{frame}
	\textbf{Probaremos}: Toda función $\Sigma$-computable es $\Sigma$-Turing computable.

	\vspace{3mm}
	\PN Si $f$ es una función $\Sigma$-mixta que es computada por un programa $\mathcal{P} \in
	\mathrm{Pro}^{\Sigma}$, entonces existe una máquina de Turing determinística con unit $M$ la cual computa a $f$.

	\vspace{3mm}
	\textbf{Definición}: Dado $\mathcal{P} \in \mathrm{Pro}^{\Sigma}$, definamos:
	\begin{eqnarray*}
		N(\mathcal{P}) &=& \text{menor k} \in \mathbb{N}\ \text{tal que las variables que ocurren en } \mathcal{P} \\
		&& \text{están todas en la lista N}1, \dotsc, \mathrm{N}\bar{k}, \mathrm{P}1, \dotsc, \mathrm{P}\bar{k}
	\end{eqnarray*}

	\PN \underline{Ejemplo}: Sea $\Sigma = \{\&,\#\}$, si $\mathcal{P}$ es el siguiente programa:
	\begin{equation*}
		\begin{array}{ll}
			\mathrm{L}1 & \mathrm{N}4\leftarrow \mathrm{N}4+1 \\
			& \mathrm{P}1\leftarrow \mathrm{P}1.\& \\
			& \mathrm{IF\ N}1\neq 0\ \mathrm{GOTO}\;\mathrm{L}1
		\end{array}
	\end{equation*}
	entonces tenemos $N(\mathcal{P})=4$
\end{frame}
