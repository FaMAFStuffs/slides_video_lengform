\begin{frame}
	\begin{itemize}
		\item Armaremos la máquina simuladora como concatenación de máquinas. Para esto, a continuación describiremos, para
		los distintos tipos de instrucciones posibles de $\mathcal{P}$, sus respectivas máquinas asociadas.

		\item Asumiremos que en $\mathcal{P}$ no hay instrucciones de la forma $\mathrm{GOTO} \ \mathrm{L}\bar{m}$, ni de la
		forma $\mathrm{L}\bar{n} \ \mathrm{GOTO} \ \mathrm{L}\bar{m}$.

		\item En esta etapa solo describiremos que propiedades tendrá que tener cada máquina simuladora de cada tipo posible
		de instrucción, y más adelante mostraremos como pueden ser construídas efectivamente dichas máquinas.

		\item Todas las máquinas descriptas tendrán:
		 	\begin{itemize}
				\item $\shortmid$ como unit
				\item $B$ como blanco
				\item $\Sigma$ como su alfabeto terminal
				\item su alfabeto mayor será $\Gamma = \Sigma \cup \{B,\shortmid\} \cup \{\tilde{a}: a \in \Sigma \cup
					\{\shortmid\}\}$.
				\item uno o dos estados finales con la siguiente propiedad:
					\vspace{3mm}
					\begin{center}
						Si $q$ es un estado final $\Rightarrow \delta(q,\sigma) = \emptyset$, para cada $\sigma \in \Gamma$
					\end{center}
			\end{itemize}
	\end{itemize}

\end{frame}
