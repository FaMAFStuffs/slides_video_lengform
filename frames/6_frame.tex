\begin{frame}
	\PN Armaremos la máquina simuladora como concatenación de máquinas, cada una de las cuales simularán, vía la
	representación anterior, el funcionamiento de las distintas instrucciones de $\mathcal{P}$. Para esto, a
	continuacion describiremos para los distintos tipos de instrucciones
	posibles de $\mathcal{P}$, sus respectivas maquinas asociadas. Asumiremos
	que en $\mathcal{P}$ no hay instrucciones de la forma $\mathrm{GOTO}\;%
	\mathrm{L}\bar{m}$ ni de la forma $\mathrm{L}\bar{n}\ \mathrm{GOTO}\;\mathrm{%
	L}\bar{m}$. Esto simplificara un poco la costruccion de la maquina
	simuladora y de hecho lo podemos hacer ya que toda funcion $\Sigma $%
	-computable puede ser computada por un programa sin este tipo de
	instrucciones, tal como lo veremos en un lema mas adelante.

	En esta etapa solo describiremos que propiedades tendra que tener cada
	maquina simuladora de cada tipo posible de instruccion, y mas adelante
	mostraremos como pueden ser construidas efectivamente dichas maquinas. Todas
	las maquinas descriptas tendran a $\shortmid $ como unit y a $B$ como
	blanco, tendran a $\Sigma $ como su alfabeto terminal y su alfabeto mayor
	sera $\Gamma =\Sigma \cup \{B,\shortmid \}\cup \{\tilde{a}:a\in \Sigma \cup
	\{\shortmid \}\}$. Ademas tendran uno o dos estados finales con la propiedad
	de que si $q$ es un estado final, entonces $\delta (q,\sigma )=\emptyset $,
	para cada $\sigma \in \Gamma $. Esta propiedad es importante ya que nos
	permitira concatenar pares de dichas maquinas identificando algun estado
	final de la primera con el inicial de la segunda.
\end{frame}
