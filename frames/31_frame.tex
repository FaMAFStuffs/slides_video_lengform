\begin{frame}
  \begin{block}{}
    \PN Cuando $n=0$ debemos interpretar que $D_{0} = \left(\{q_{0},q_{f}\}, \Gamma, \Sigma, \delta, q_{0}, B,
    \shortmid, \{q_{f}\}\right)$, con $\delta(q_{0},B) = \{(q_{f},B,K)\}$ y $\delta = \emptyset$ en cualquier otro caso.

    \PN Nótese que $M_{1}$ cumple que, para cada $(\vec{x},\vec{\alpha}) \in \omega^{n} \times \Sigma^{\ast m}$
    \normLetter
    \begin{equation*}
      \left\lfloor q_{0} B \shortmid^{x_{1}} B \dotsc B \shortmid^{x_{n}} B \alpha_{1} B \dotsc B \alpha_{m} B
      \right\rfloor \overset{\ast}{\vdash} \left\lfloor q_{f} B \shortmid^{x_{1}} B \dotsc B \shortmid^{x_{n}} B^{k-n}B
      \alpha_{1} B \dotsc B \alpha_{m} B \right\rfloor
    \end{equation*}
    \sizeOfLetterThird
    \PN Nótese que en la confección de $M_{1}$, para el caso $m>0$ podriamos haber usado directamente la $TD_{m}$ en
    lugar de usar $TD_{m}$.
  \end{block}
\end{frame}
