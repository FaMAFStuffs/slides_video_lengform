\begin{frame}
  \begin{proof}
  Supongamos $O=\Sigma ^{\ast }$. Por el Lema \ref{sinGOTO} existe $\mathcal{P}%
  \in \mathrm{Pro}^{\Sigma }$ el cual computa $f$ y tiene las propiedades (1)
  y (2). Sea $k=\max \{n,m,N(\mathcal{P})\}$ y sea $M_{sim}$ la maquina de
  Turing con unit que simula a $\mathcal{P}$ respecto de $k$. Como puede
  observarse, la maquina $M_{sim}$, no necesariamente computara a $f$. Sea $%
  M_{1}$ la maquina siguiente

  Figura 9

  (Cuando $n=0$ debemos interpretar que $D_{0}=\left( \{q_{0},q_{f}\},\Gamma
  ,\Sigma ,\delta ,q_{0},B,\shortmid ,\{q_{f}\}\right) $, con $\delta
  (q_{0},B)=\{(q_{f},B,K)\}$ y $\delta =\emptyset $ en cualquier otro caso).

  Notese que $M_{1}$ cumple que para cada $(\vec{x},\vec{\alpha})\in \omega
  ^{n}\times \Sigma ^{\ast m}$,%
  \begin{equation*}
  \left\lfloor q_{0}B\shortmid ^{x_{1}}B...B\shortmid ^{x_{n}}B\alpha
  _{1}B...B\alpha _{m}B\right\rfloor \overset{\ast }{\vdash }\left\lfloor
  q_{f}B\shortmid ^{x_{1}}B...B\shortmid ^{x_{n}}B^{k-n}B\alpha
  _{1}B...B\alpha _{m}B\right\rfloor
  \end{equation*}%
  Notese que en la confeccion de $M_{1}$, para el caso $m>0$ podriamos haber
  usado directamente la $TD_{m}$ en lugar de usar $TD_{m}$.

  Sea $M_{2}$ la siguiente maquina

  Figura 10

  Notese que $M_{2}$ cumple que para cada $\alpha \in \Sigma ^{\ast }$,%
  \begin{equation*}
  \left\lfloor q_{0}B^{k+1}\alpha \right\rfloor \overset{\ast }{\vdash }%
  \left\lfloor q_{f}B\alpha \right\rfloor
  \end{equation*}%
  Sea $M$ la maquina dada por el siguiente diagrama

  Figura 11

  A continuacion veremos que $M$ computa a $f$. Supongamos que $(\vec{x},\vec{%
  \alpha})\in (\omega ^{n}\times \Sigma ^{\ast m})-D_{f}$. Deberemos ver que $M
  $ no termina partiendo de

  \begin{enumerate}
  \item[(*)] $\left\lfloor q_{0}B\shortmid ^{x_{1}}B...B\shortmid
  ^{x_{n}}B\alpha _{1}B...B\alpha _{m}B\right\rfloor $
  \end{enumerate}

  Primero notemos que, ya que $\mathcal{P}$ computa a $f$, tenemos que $%
  \mathcal{P}$ no termina partiendo de $\left\Vert x_{1},...,x_{n},\alpha
  _{1},...,\alpha _{m}\right\Vert $ por lo cual $\mathcal{P}$ no termina
  partiendo de%
  \begin{equation*}
  \left\Vert x_{1},...,x_{n},\overset{k-n}{\overbrace{0,...,0}},\alpha
  _{1},...,\alpha _{m},\overset{k-m}{\overbrace{\varepsilon ,...,\varepsilon }}%
  \right\Vert
  \end{equation*}%
  lo cual implica (Lema \ref{simulacion}) que

  \begin{enumerate}
  \item[(**)] $M_{sim}$ no termina partiendo de $\left\lfloor q_{0}B\shortmid
  ^{x_{1}}B...B\shortmid ^{x_{n}}B^{k-n}B\alpha _{1}B...B\alpha
  _{m}B\right\rfloor $
  \end{enumerate}

  Ahora notese que si hacemos funcionar a $M$ desde la descripcion instantanea
  dada en (*), llegaremos (via la copia de $M_{1}$ dentro de $M$)
  indefectiblemente (ya que $M$ es deterministica) a la siguiente descripcion
  instantanea%
  \begin{equation*}
  \left\lfloor q_{2}B\shortmid ^{x_{1}}B...B\shortmid ^{x_{n}}B^{k-n}B\alpha
  _{1}B...B\alpha _{m}B\right\rfloor
  \end{equation*}%
  Luego entonces (**) nos dice que al seguir trabajando $M$ (ahora via la
  copia de $M_{sim}$ dentro de $M$), la maquina $M$ nunca terminara.

  Para terminar de ver que $M$ computa a $f$, tomemos $(\vec{x},\vec{\alpha}%
  )\in D_{f}$ y veamos que%
  \begin{equation*}
  \left\lfloor q_{0}B\shortmid ^{x_{1}}B...B\shortmid ^{x_{n}}B\alpha
  _{1}B...B\alpha _{m}B\right\rfloor \overset{\ast }{\underset{M}{\vdash }}%
  \left\lfloor q_{5}Bf(\vec{x},\vec{\alpha})\right\rfloor
  \end{equation*}%
  y que la maquina $M$ se detiene en $\left\lfloor q_{5}Bf(\vec{x},\vec{\alpha}%
  )\right\rfloor $. La maquina $M$ se detiene en $\left\lfloor q_{5}Bf(\vec{x},%
  \vec{\alpha})\right\rfloor $ ya que $q_{5}$ es el estado final de una copia
  de $M_{2}$ y por lo tanto no sale ninguna flecha desde el. Ya que $\mathcal{P%
  }$ computa a $f$ y tiene la propiedad (2) del Lema \ref{sinGOTO}, tenemos
  que $\mathcal{P}$ termina partiendo de $\left\Vert x_{1},...,x_{n},\alpha
  _{1},...,\alpha _{m}\right\Vert $ y llega al estado $\left\Vert f(\vec{x},%
  \vec{\alpha})\right\Vert $, o lo que es lo mismo, $\mathcal{P}$ termina
  partiendo de%
  \begin{equation*}
  \left\Vert x_{1},...,x_{n},\overset{k-n}{\overbrace{0,...,0}},\alpha
  _{1},...,\alpha _{m},\overset{k-m}{\overbrace{\varepsilon ,...,\varepsilon }}%
  \right\Vert
  \end{equation*}%
  y llega al estado%
  \begin{equation*}
  \left\Vert \overset{k}{\overbrace{0,...,0}},f(\vec{x},\vec{\alpha}),\overset{%
  k-1}{\overbrace{\varepsilon ,...,\varepsilon }}\right\Vert
  \end{equation*}%
  Pero entonces el Lema \ref{simulacion} nos dice que

  \begin{enumerate}
  \item[(***)] $\left\lfloor q_{0}B\shortmid ^{x_{1}}B...B\shortmid
  ^{x_{n}}B^{k-n}B\alpha _{1}B...B\alpha _{m}B\right\rfloor \overset{\ast }{%
  \underset{M_{sim}}{\vdash }}\left\lfloor q_{f}B^{k+1}f(\vec{x},\vec{\alpha}%
  )\right\rfloor $
  \end{enumerate}

  Como ya lo vimos, si hacemos funcionar a $M$ desde $\left\lfloor
  q_{0}B\shortmid ^{x_{1}}B...B\shortmid ^{x_{n}}B\alpha _{1}B...B\alpha
  _{m}B\right\rfloor $, llegaremos (via la copia de $M_{1}$ dentro de $M$)
  indefectiblemente a la siguiente descripcion instantanea%
  \begin{equation*}
  \left\lfloor q_{2}B\shortmid ^{x_{1}}B...B\shortmid ^{x_{n}}B^{k-n}B\alpha
  _{1}B...B\alpha _{m}B\right\rfloor
  \end{equation*}%
  Luego (***) nos dice que, via la copia de $M_{sim}$ dentro de $M$,
  llegaremos a $\left\lfloor q_{3}B^{k+1}f(\vec{x},\vec{\alpha})\right\rfloor $
  e inmediatamente a $\left\lfloor q_{4}B^{k+1}f(\vec{x},\vec{\alpha}%
  )\right\rfloor $. Finalmente, via la copia de $M_{2}$ dentro de $M$,
  llegaremos a $\left\lfloor q_{5}Bf(\vec{x},\vec{\alpha})\right\rfloor $, lo
  cual termina de demostrar que $M$ computa a $f$
  \end{proof}
\end{frame}
