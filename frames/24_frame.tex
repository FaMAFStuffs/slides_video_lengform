\begin{frame}
  Analogamente, para $j\geq 1$, sea $TI_{j}$ una maquina tal que%
  \begin{equation*}
  \begin{array}{ccc}
  \alpha B\sigma \gamma  & \overset{\ast }{\vdash } & \alpha B\gamma  \\
  \uparrow \  &  & \uparrow  \\
  q_{0}\ \  &  & q_{f}%
  \end{array}%
  \end{equation*}%
  cada vez que $\alpha \in \Gamma ^{\ast }$, $\sigma \in \Gamma $ y $\gamma $
  tiene exactamente $j$ ocurrencias de $B$. Es decir la maquina $TI_{j}$ corre
  un espacio a la izquierda todo el segmaneto $\gamma $ (por lo cual en el
  lugar de $\sigma $ queda el primer simbolo de $\gamma $). Dejamos al lector
  la construccion de por ejemplo $TI_{3}$ para $\Sigma =\{\&,\#\}$

  Teniendo las maquinas auxiliares antes definidas podemos combinarlas para
  obtener las maquinas simuladoras de instrucciones. Por ejemplo $M_{i,k}^{a}$
  puede ser la siguiente maquina
\end{frame}
