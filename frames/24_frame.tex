\begin{frame}
  \PN Análogamente, para $j \geq 1$, sea $TI_{j}$ una máquina tal que
  \begin{equation*}
    \begin{array}{ccc}
    \alpha B\sigma \gamma  & \overset{\ast }{\vdash } & \alpha B\gamma  \\
    \uparrow \  &  & \uparrow  \\
    q_{0}\ \  &  & q_{f}%
    \end{array}%
  \end{equation*}%
  \PN cada vez que $\alpha \in \Gamma^{\ast}, \sigma \in \Gamma$ y $\gamma$ tiene exactamente $j$ ocurrencias de $B$.
  Dejamos al lector la construcción de, por ejemplo, $TI_{3}$ para $\Sigma = \{\&,\#\}$.

  \vspace{7mm}
  \begin{alertblock}
  \PN Teniendo las máquinas auxiliares antes definidas podemos combinarlas para obtener las máquinas simuladoras de
  instrucciones.
\end{alertblock}
\end{frame}
