\begin{frame}
  \begin{alertblock}{Lema}
    \PN Sea $\mathcal{P} \in \mathrm{Pro}^{\Sigma}$ y sea $k \geq N(\mathcal{P})$. Supongamos que en $\mathcal{P}$ no
    hay instrucciones de la forma $\mathrm{GOTO} \ \mathrm{L}\bar{m}$ ni de la forma $\mathrm{L}\bar{n} \ \mathrm{GOTO}
    \ \mathrm{L}\bar{m}$. Sean:
    \begin{itemize}
      \item Para cada $a \in \Sigma \cup \{\shortmid\}$, sea $\tilde{a}$ un nuevo símbolo
      \item $\Gamma = \Sigma \cup \{B, \shortmid\} \cup \{\tilde{a}: a \in \Sigma \cup \{\shortmid\}\}$
    \end{itemize}
    \PN entonces existe una máquina de Turing determinística con unit $M = \left(Q, \Gamma, \Sigma, \delta, q_{0}, B,
    \shortmid,\{q_{f}\}\right)$, la cual satisface:
    \begin{enumerate}[1)]
      \item $\delta (q_{f},\sigma) = \emptyset$, para cada $\sigma \in \Gamma$.

      \item Cualesquiera sean $x_{1}, \dotsc, x_{k} \in \omega$ y $\alpha_{1}, \dotsc, \alpha_{k} \in \SIGMA$, el
      programa $\mathcal{P}$ se detiene partiendo del estado
      \begin{equation*}
        \lVert x_{1}, \dotsc, x_{k}, \alpha_{1}, \dotsc, \alpha_{k} \rVert
      \end{equation*}
      \PN si y solo si $M$ se detiene partiendo de la descripción instantánea
      \begin{equation*}
        \left\lfloor q_{0} B \shortmid^{x_{1}} B \dotsc B \shortmid^{x_{k}} B \alpha_{1} B \dotsc B \alpha_{k} B
        \right\rfloor
      \end{equation*}
    \end{enumerate}
  \end{alertblock}
\end{frame}
