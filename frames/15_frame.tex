\begin{frame}
  \PN Veamos con un ejemplo como $M_{sim}$ simula a $\mathcal{P}$. Supongamos que corremos $\mathcal{P}$ desde el estado
  \begin{equation*}
    \lVert 2, 1, 0, 5, 3, \#\&\#\#, \varepsilon, \&\&, \#\&, \# \rVert
  \end{equation*}
  \PN Tendremos entonces la siguiente sucesión de descripciones instantáneas:
  \begin{eqnarray*}
    && (1, \lVert 2, 1, 0, 5, 3, \#\&\#\#, \varepsilon, \&\&, \#\&, \# \rVert) \\[5pt]
    && (2, \lVert 2, 1, 0, 6, 3, \#\&\#\#, \varepsilon, \&\&, \#\&, \# \rVert) \\[5pt]
    && (3, \lVert 2, 1, 0, 6, 3, \&\#\#, \varepsilon, \&\&, \#\&, \# \rVert) \\[5pt]
    && (1, \lVert 2, 1, 0, 6, 3, \&\#\#, \varepsilon, \&\&, \#\&, \# \rVert) \\[5pt]
    && (2, \lVert 2, 1, 0, 7, 3, \&\#\#, \varepsilon, \&\&, \#\&, \# \rVert) \\[5pt]
    && (3, \lVert 2, 1, 0, 7, 3, \#\#, \varepsilon, \&\&, \#\&, \# \rVert) \\[5pt]
    && (4, \lVert 2, 1, 0, 7, 3, \#\#, \varepsilon, \&\&, \#\&, \# \rVert) \\[5pt]
    && (5, \lVert 2, 1, 0, 7, 3, \#\#, \varepsilon, \&\&\#, \#\&, \# \rVert)
  \end{eqnarray*}
\end{frame}
