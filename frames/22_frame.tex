\begin{frame}
  \PN $I_{j}$ cumplirá que:
  \begin{equation*}
    \begin{array}{lcr}
      \alpha B \beta_{j} B \dotsc B \beta_{2} B \beta_{1} B \gamma &\overset{\ast}{\vdash}& \alpha B \beta_{j} B \dotsc
        B \beta_{2} B \beta_{1} B \gamma \\
      \ \ \ \ \ \ \ \ \ \ \ \ \ \ \ \ \ \ \ \ \ \ \ \;\; \uparrow && \uparrow \ \ \ \ \ \ \ \ \ \ \ \ \ \ \ \ \ \ \ \ \
        \ \ \  \\
      \ \ \ \ \ \ \ \ \ \ \ \ \ \ \ \ \ \ \ \ \ \ \ \ q_{0} && q_{f} \ \ \ \ \ \ \ \ \ \ \ \ \ \ \ \ \ \ \ \ \ \ \
    \end{array}
  \end{equation*}
  \PN siempre que $\alpha, \gamma \in \Gamma^{\ast}, \beta_{1}, \dotsc, \beta_{j} \in (\Gamma - \{B\})^{\ast}$. Dejamos
  al lector la manufactura de esta máquina.

  \vspace{3mm}
  \PN Para $j \geq 1$, sea $TD_{j}$ una máquina con un solo estado final $q_{f}$ y tal que:
  \begin{equation*}
    \begin{array}{ccc}
      \alpha B \gamma &\overset{\ast}{\vdash}& \alpha BB \gamma \\
      \uparrow  && \uparrow \ \ \\
      q_{0} &  & q_{f} \ \
    \end{array}
  \end{equation*}
  \PN cada vez que $\alpha, \gamma \in \Gamma^{\ast}$ y $\gamma$ tiene exactamente $j$ ocurrencias de $B$.
\end{frame}
