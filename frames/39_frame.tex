\begin{frame}
  \begin{block}{}
    \PN Como ya lo vimos, si hacemos funcionar a $M$ desde:
    \begin{equation*}
      \left\lfloor q_{0} B \shortmid^{x_{1}} B \dotsc B \shortmid^{x_{n}} B \alpha_{1} B \dotsc B \alpha_{m} B
      \right\rfloor
    \end{equation*}
    \PN llegaremos, vía la copia de $M_{1}$ dentro de $M$, indefectiblemente a la siguiente descripción instantánea
    \begin{equation*}
      \left\lfloor q_{2} B \shortmid^{x_{1}} B \dotsc B \shortmid^{x_{n}} B^{k-n}B \alpha_{1} B \dotsc B \alpha_{m} B
      \right\rfloor
    \end{equation*}

    \PN Luego, \textcolor{red}{(***)} nos dice que, vía la copia de $M_{sim}$ dentro de $M$, llegaremos a $\left\lfloor
    q_{3} B^{k+1} f(\vec{x},\vec{\alpha}) \right\rfloor$ e inmediatamente a $\left\lfloor q_{4} B^{k+1} f(\vec{x},
    \vec{\alpha}) \right\rfloor$.

    \vspace{3mm}
    \PN Finalmente, vía la copia de $M_{2}$ dentro de $M$, llegaremos a $\left\lfloor q_{5} B f(\vec{x},\vec{\alpha})
    \right\rfloor$, lo cual termina de demostrar que $M$ computa a $f$.
  \end{block}
\end{frame}
