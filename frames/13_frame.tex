\begin{frame}
  Veamos con un ejemplo como $M_{sim}$ simula a $\mathcal{P}$. Supongamos que
  corremos $\mathcal{P}$ desde el estado%
  \begin{equation*}
  \left\Vert 2,1,0,5,3,\#\&\#\#,\varepsilon ,\&\&,\#\&,\#\right\Vert
  \end{equation*}%
  Tendremos entonces la siguiente sucesion de descripciones instantaneas:%
  \begin{eqnarray*}
  &&(1,\left\Vert 2,1,0,5,3,\#\&\#\#,\varepsilon ,\&\&,\#\&,\#\right\Vert ) \\
  && \\
  &&(2,\left\Vert 2,1,0,6,3,\#\&\#\#,\varepsilon ,\&\&,\#\&,\#\right\Vert ) \\
  && \\
  &&(3,\left\Vert 2,1,0,6,3,\&\#\#,\varepsilon ,\&\&,\#\&,\#\right\Vert ) \\
  && \\
  &&(1,\left\Vert 2,1,0,6,3,\&\#\#,\varepsilon ,\&\&,\#\&,\#\right\Vert ) \\
  && \\
  &&(2,\left\Vert 2,1,0,7,3,\&\#\#,\varepsilon ,\&\&,\#\&,\#\right\Vert ) \\
  && \\
  &&(3,\left\Vert 2,1,0,7,3,\#\#,\varepsilon ,\&\&,\#\&,\#\right\Vert ) \\
  && \\
  &&(4,\left\Vert 2,1,0,7,3,\#\#,\varepsilon ,\&\&,\#\&,\#\right\Vert ) \\
  && \\
  &&(5,\left\Vert 2,1,0,7,3,\#\#,\varepsilon ,\&\&\#,\#\&,\#\right\Vert )
  \end{eqnarray*}

\end{frame}
