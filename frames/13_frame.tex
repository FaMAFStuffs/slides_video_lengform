\begin{frame}
  \begin{example}
    \PN Sea $\Sigma = \{\&,\#\}$ y sea $\mathcal{P}$ el siguiente programa:
    \begin{equation*}
      \begin{array}{ll}
        \mathrm{L}3 & \mathrm{N}4\leftarrow \mathrm{N}4+1 \\
        & \mathrm{P}1\leftarrow \ ^{\curvearrowright }\mathrm{P}1 \\
        & \mathrm{IF\ P}1\ \mathrm{BEGINS\ }\&\ \mathrm{GOTO}\;\mathrm{L}3 \\
        & \mathrm{P}3\leftarrow \mathrm{P}3.\#
      \end{array}
    \end{equation*}
    \PN Tomemos $k=5$, es claro que $k \geq N(\mathcal{P}) = 4$. A la máquina que simulará a $\mathcal{P}$ respecto de
    $k$, la llamaremos $M_{sim}$ y será la siguiente:
\end{example}
\end{frame}
