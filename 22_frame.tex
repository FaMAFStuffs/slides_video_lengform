\begin{frame}
  Para $j\geq 1$, sea $TD_{j}$ una maquina con un solo estado final $q_{f}$ y
  tal que%
  \begin{equation*}
  \begin{array}{ccc}
  \alpha B\gamma  & \overset{\ast }{\vdash } & \alpha BB\gamma  \\
  \uparrow  &  & \uparrow \ \  \\
  q_{0} &  & q_{f}\ \
  \end{array}%
  \end{equation*}%
  cada vez que $\alpha ,\gamma \in \Gamma ^{\ast }$ y $\gamma $ tiene
  exactamente $j$ ocurrencias de $B$. Es decir la maquina $TD_{j}$ corre un
  espacio a la derecha todo el segmento $\gamma $ y agrega un blanco en el
  espacio que se genera a la izquierda. Por ejemplo, para el caso de $\Sigma
  =\{\&,\#\}$ podemos tomar $TD_{3}$ igual a la siguiente maquina
\end{frame}
