\begin{frame}
  \noindent Ya que la maquina $M_{i}$ puede tener uno o dos estados finales,
  la representaremos de la siguiente manera

  \bigskip

  \bigskip

  \bigskip

  \bigskip

  Figura 5

  \bigskip

  \bigskip

  \bigskip

  \noindent entendiendo que en el caso en que $M_{i}$ tiene un solo estado
  final, este esta representado por el circulo de abajo a la izquierda y en el
  caso en que $M_{i}$ tiene dos estados finales, el estado final representado
  con lineas punteadas (es decir, el de la derecha) corresponde al estado $%
  q_{si}$ y el otro al estado $q_{no}$.

\end{frame}
