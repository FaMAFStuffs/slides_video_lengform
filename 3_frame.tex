\begin{frame}
	Sea $\mathcal{P}$ un programa y sea $k$ fijo y mayor o igual a $N(\mathcal{P})$. A continuación describiremos como puede construirse una maquina de
	Turing la cual simulara a $\mathcal{P}$. La construccion de la maquina
	simuladora dependera de $\mathcal{P}$ y de $k$. Notese que cuando $\mathcal{P%
	}$ se corre desde algun estado de la forma%
	\begin{equation*}
	\left\Vert x_{1},...,x_{k},\alpha _{1},...,\alpha _{k}\right\Vert
	\end{equation*}%
	los sucesivos estados por los que va pasando son todos de la forma

	\begin{equation*}
	\left\Vert y_{1},...,y_{k},\beta _{1},...,\beta _{k}\right\Vert
	\end{equation*}%
	es decir en todos ellos las variables con indice mayor que $k$ valen $0$ o $%
	\varepsilon $. La razon es simple: ya que en $\mathcal{P}$ no figuran las
	variables%
	\begin{eqnarray*}
	&&\mathrm{N}\overline{k+1},\mathrm{N}\overline{k+2},... \\
	&&\mathrm{P}\overline{k+1},\mathrm{P}\overline{k+2},...
	\end{eqnarray*}%
	estas variables quedan con valores $0$ y $\varepsilon $, respectivamente a
	lo largo de toda la computacion.
\end{frame}
