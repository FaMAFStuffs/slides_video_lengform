\begin{frame}

  En lo que sigue probaremos que el paradigma computacional de Turing
es por lo menos tan expresivo como el paradigma imperativo dado por el
lenguaje $\mathcal{S}^{\Sigma }$, es decir probaremos que toda funcion $%
\Sigma $-computable es $\Sigma $-Turing computable\textit{.} Antes un lema

\bigskip

\begin{lemma}
\label{sinGOTO}Si $f:D_{f}\subseteq \omega ^{n}\times \Sigma ^{\ast
m}\rightarrow \Sigma ^{\ast }$ es $\Sigma $-computable, entonces hay un
programa $\mathcal{Q}$ el cual computa a $f$ y el cual cumple con las
siguientes propiedades

\begin{enumerate}
\item[(1)] En $\mathcal{Q}$ no hay instrucciones de la forma $\mathrm{GOTO}\;%
\mathrm{L}\bar{\imath}$ ni de la forma $\mathrm{L}\bar{j}\ \mathrm{GOTO}\;%
\mathrm{L}\bar{\imath}$

\item[(2)] Cuando $\mathcal{Q}$ termina partiendo de un estado cualquiera
dado, el estado alcansado es tal que las variables numericas tienen todas el
valor $0$ y las alfabeticas tienen todas exepto $\mathrm{P}1$ el valor $%
\varepsilon $.
\end{enumerate}
\end{lemma}

\end{frame}
