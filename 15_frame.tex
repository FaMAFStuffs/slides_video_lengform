\begin{frame}
  Ahora describiremos en general como puede armarse la maquina
  simuladora de $\mathcal{P}$, respecto de $k$. Supongamos que $\mathcal{P}%
  =I_{1}...I_{n}$. Para cada $i=1,...,n$, llamaremos $M_{i}$ a la maquina que
  simulara el efecto que produce la instruccion $I_{i}$, es decir tomemos
\end{frame}
