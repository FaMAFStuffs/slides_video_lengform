\begin{frame}
	Necesitaremos tener alguna manera de representar en la cinta los diferentes
	estados por los cuales se va pasando, a medida que corremos a $\mathcal{P}$.
	Esto lo haremos de la siguiente forma: al estado

	\begin{equation*}
	\left\Vert x_{1},...,x_{k},\alpha _{1},...,\alpha _{k}\right\Vert
	\end{equation*}%
	lo representaremos en la cinta de la siguiente manera%
	\begin{equation*}
	B\shortmid ^{x_{1}}...B\shortmid ^{x_{k}}B\alpha _{1}...B\alpha _{k}BBBB....
	\end{equation*}%
	Por ejemplo consideremos el programa $\mathcal{P}$ mostrado recien y fijemos
	$k=6$. Entonces al estado%
	\begin{equation*}
	\left\Vert 3,2,5,0,4,2,\&,\&\&,\varepsilon ,\#\&,\#,\#\#\#\right\Vert
	\end{equation*}%
	lo representaremos en la cinta de la siguiente manera%
	\begin{equation*}
	B\shortmid \shortmid \shortmid B\shortmid \shortmid B\mathrm{\shortmid
	\shortmid \shortmid \shortmid \shortmid }BB\shortmid \shortmid \shortmid
	\shortmid B\shortmid \shortmid B\&B\&\&BB\#\&B\#B\#\#\#BBBBB....
	\end{equation*}%
\end{frame}
